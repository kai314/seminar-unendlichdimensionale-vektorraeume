% !TeX spellcheck = de_DE
\documentclass[AERbeamer%              style
%,optEnglish%            language
,handout%               deactivate animation
%              ,optBiber%               bibliography tool
%              ,optBibstyleAlphabetic%
,optBeamerClassicFormat% 4:3 format
%,optBeamerWideFormat%   16:9 format
,optLeftEquations   % left align equations
]{AERlatex}
\usepackage{blkarray}%
\usepackage{enumitem}%
%\setlength{\mathindent}{0pt}%
%\setlength{\baselineskip}{20pt}
%\setbeameroption{show notes}% Show all notes
%
% Set paths
\graphicspath{{figures/}}%
%\addbibresource{source/literature.bib}%
%
% set meta data
\title{Lineare Abbildungen}%
\subtitle{Ferienseminar: Unendlichdimensionale Vektorräume}% (optional)
\author{Kai Eberl}% (optional)
\date{\AERutilsDate{7}{9}{2022}}% (optional)
%\institute{Institute}% (optional)
%
% Setup of header and footer
\AERbeamerSetupHeader{\AERlayoutHeaderCDTUMLogoOnly}%
%\AERbeamerSetupHeader{\AERlayoutHeaderCDTUM}%
%\AERbeamerSetupHeader{\AERlayoutHeaderCDDepartment}%
%\AERbeamerSetupHeader{\AERlayoutHeaderCDChair}%
\AERbeamerSetupFooterCD%
%\AERbeamerSetupFooterSlideNumberOnly%
%
\begin{document}%
%
% Start with titlepage
    \AERbeamerTitlePageDefault%
%
    \begin{frame}{Was ist eine lineare Abbildung?}%
        \begin{Definition}
            Eine Abbildung $~f: V \rightarrow W$ zwischen $\mathbb{K}$-Vektorräumen $V$ und $W$ heißt \emph{lineare Abbildung} (Homomorphismus), wenn $\forall v, ~ w \in V, ~ \lambda \in \mathbb{K}$ gilt: \pause
            \begin{enumerate}
                [label=$(\roman*)$, leftmargin=2em]
                \item $f(v+w) = f(v) + f(w)$ \hspace{2em} Additivität \pause
                \item $f(\lambda v) = \lambda f(v)$ \hspace{6em} Homogenität
            \end{enumerate}
        \end{Definition}
    \end{frame}%
%
    \begin{frame}{Darstellung als Matrix}
        Sei $f: \mathbb{K}^n \rightarrow \mathbb{K}^m$ eine lineare Abbildung und $B=(b_1, \dots , b_n)$ eine Basis von $\mathbb{K}^n$.
        \begin{flalign*}
            v = \sum_{i=1}^n \lambda_i b_i && \pause
        \end{flalign*}
        Wegen Additivität und Homogenität:
        \begin{flalign*}
            f(v) = f(\lambda_1 b_1 + \dots + \lambda_n b_n) = \lambda_1 ~ f(b_1) + \dots + \lambda_n ~ f(b_n) &&
        \end{flalign*}
        $\Rightarrow ~ f(v)$ ist eindeutig durch die Bilder der Basisvektoren bestimmt.
    \end{frame}
%
    \begin{frame}{Darstellung als Matrix}
        \begin{block}{Prinzip der linearen Fortsetzung}
            Seien $V$, $W$ Vektorräume in $\mathbb{K}$ und $B=(b_1, \dots, b_n)$ eine Basis von $V$. \\
            Sei nun $A:\left\{\begin{array}{l}
                                  B \rightarrow W \\ b \mapsto A(b)
            \end{array} \right.$ eine lineare Abbildung.                                             \\
            Dann gibt es genau eine lineare Abbildung $~ f: ~ V \rightarrow W$ mit $~ f|_B = A$. \\
            $f$ ist die lineare Fortsetzung von $A$ auf $V$.
        \end{block}
    \end{frame}
%
    \begin{frame}{Beweis}
        \noindent
        \begin{flalign*}
            f(v) &\coloneqq\lambda_1 A\left(b_1\right)+\cdots+\lambda_n A\left(b_n\right) \in W, \quad \text{wobei}
            ~ v = \lambda_1 b_1+\cdots+\lambda_n b_n \in V &&
        \end{flalign*}
        \vspace{0.5em}
        \textbf{Linearität:}
        \begin{flalign*}
            \text{sei} ~v=\sum_{i=1}^n \lambda_i b_i, \quad w=\sum_{i=1}^n \mu_i c_i, \quad \lambda_i, \mu_i \in \mathbb{K} && \pause
        \end{flalign*}
        \begin{flalign*}
            f(v+w) &=f\left(\sum_{i=1}^n \lambda_i b_i+\sum_{i=1}^n \mu_i c_i\right)=A\left(\sum_{i=1}^n \lambda_i b_i+\sum_{i=1}^n \mu_i c_i\right) && \\ \pause
            &=A\left(\sum_{i=1}^n \lambda_i b_i\right)+A\left(\sum_{i=1}^n \mu_i c_i\right)=A(v)+A(w) = f(v)+f(w) \\ \pause \pause
            f\left(\lambda v\right) &=f\left(\lambda \sum_{i=1}^n \lambda_i b_i\right)=A\left(\lambda \sum_{i=1}^n \lambda_i b_i\right)=\lambda A\left(\sum_{i=1}^n \lambda_i b_i\right)=\lambda A(v)=\lambda f(v) \pause
        \end{flalign*}
        $\Rightarrow ~ f ~$ ist linear
    \end{frame}
%
    \begin{frame}{Beweis}
        \textbf{Eindeutigkeit:} \\
        Für zwei lineare Fortsetzungen $f$, $g$ von $A$ gilt:
        \begin{flalign*}
            f(v) &=f\left(\lambda_1 b_1+\cdots+\lambda_n b_n\right) && \\ \pause
            &=\lambda_1 A\left(b_1\right)+\cdots+\lambda_n A\left(b_n\right) && \\ \pause
            &=g\left(\lambda_1 b_1+\cdots+\lambda_n b_n\right) && \\ \pause
            &=g(v) && \\ \pause
            \Rightarrow ~ f &= g &&
        \end{flalign*}
    \end{frame}
%
%    \begin{frame}{Stetigkeit in endlichdimensionalen Vektorräumen}
%    \end{frame}
%
%    \begin{frame}{Darstellung als Matrix}
%        \begin{block}{Koordinatenvektoren bezüglich einer Basis}
%            Ist $~B=\left(b_1, \ldots, b_n\right)~$ eine geordnete Basis eines $\mathbb{K}$-Vektorraums $V$, so besitzt jedes $v \in V$ genau eine Darstellung
%            \begin{flalign*}
%                v = v_1 b_1 + \dots + v_n b_b
%            \end{flalign*}
%            mit $~v_1, \dots , v_n ~ \in ~ \mathbb{K}$. Es heißt ${ }_B v=\left(\begin{array}{c}
%                                                                                                 v_1    \\
%                                                                                                 \vdots \\
%                                                                                                 v_n
%            \end{array}\right) \in \mathbb{K}^n$ der Koordinatenvektor von $v$ bezüglich $B$.
%        \end{block}
%    \end{frame}
%
    \begin{frame}{Darstellung als Matrix}
        \begin{block}{Darstellung linearer Abbildungen bezüglich der Standardbasen}
            Zu jeder linearen Abbildung $f$ von $\mathbb{K}^n$ in $\mathbb{K}^m$ gibt es eine Matrix $~A \in \mathbb{K}^{m \times n}~$ mit $~f=f_A$. Diese Matrix $A$ ist gegeben als
            \begin{flalign*}
                A=\left(f\left(e_1\right), \ldots, f\left(e_n\right)\right) \in \mathbb{K}^{m \times n}.
            \end{flalign*}
            Die $i$-te Spalte von $A$ ist das Bild des $i$-ten Basisvektors der Standardbasis. \\
            Es gilt $f(v) = Av$. \\
            Der Koordinatenvektor von $f(v)$ ist das Produkt der Darstellungsmatrix mit dem Koordinatenvektor von $v$.
        \end{block}
    \end{frame}
%
    \begin{frame}{Stetigkeit}
        \begin{Satz}
            Für eine lineare Abbildung $L$ eines normierten Vektorraums $V$ in einen normierten Vektorraum $W$ sind folgende Eigenschaften äquivalent:
            \begin{enumerate}
                [label=$(\roman*)$, leftmargin=2em]
                \item Die Abbildung $L$ ist gleichmäßig stetig auf $V$.
                \item Die Abbildung $L$ ist stetig in $0_V$.
                \item Die Menge $\{\|Lx\|_W: ~ x \in V, ~ \|x\|_V \leq 1\}$ ist beschränkt.
            \end{enumerate}
        \end{Satz}
    \end{frame}
%
    \begin{frame}{Beweis}
        \noindent
        \begin{flalign*}
            & (i) \Rightarrow (ii): && \\ \pause
            & \text{trivial} \pause
        \end{flalign*}
        Erinnerung: \\
        Eine Funktion $~f: D \rightarrow \mathbb{C}~$ heißt \emph{gleichmäßig stetig} auf $D$, wenn
        \begin{flalign*}
            \forall \varepsilon>0 \quad \exists \delta>0: ~ \left\|f(x)-f\left(x^{\prime}\right)\right\|<\varepsilon &&
        \end{flalign*}
        für alle Punktepaare $~ x, x^{\prime} ~$ auf dem Definitionsbereich $D$ mit $~ \left\|x-x^{\prime}\right\|<\delta$.
    \end{frame}
%
    \begin{frame}{Beweis}
        \noindent
        \begin{flalign*}
            & (ii) \Rightarrow (iii): && \\ \pause
            & \text{sei} \quad \left\{\|L w\|_W: x \in V, ~ \|x \|_V \leq 1\right\} \quad \text{unbeschränkt} \\ \pause
            & \Rightarrow ~ \forall n \in \mathbb{N} \quad \exists x_n \in V: ~ \left\|x_n\right\| \leq 1, \quad \left\|L x_n\right\| \geq n \\ \pause
            & y_n \coloneqq \frac{1}{n} x_n \quad \Rightarrow \quad \left\|y_n\right\|=\frac{1}{n}\left\|x_n\right\| \leq \frac{1}{n}, \quad \left\|L y_n\right\|=\frac{1}{n}\left\|L x_n\right\| \geq 1 \\ \pause
            & \lim _{n \rightarrow \infty} y_n=0_V, \quad L 0_V=0_W \quad \text{für jede lineare Abbildung} \\ \pause
            & \text{gleichzeitig} \quad \|Ly_n\| \geq 1 \quad \forall n \in \mathbb{N} \\ \pause
            & \Rightarrow ~ L ~ \text{in} ~ 0_V ~ \text{unstetig}
        \end{flalign*}
    \end{frame}
%
    \begin{frame}{Beweis}
        \noindent
        \begin{flalign*}
            & (iii) \Rightarrow (i): && \\ \pause
            & M\coloneqq\sup \{\|L x\|: x \in V,~\|x\| \leq 1\}<\infty \\ \pause
            & \text{dafür zu zeigen:} \quad \|L x-L y\| \leq M\|x-y\| \quad \forall x, ~ y \text{ in } V \\ \pause
            & z\coloneqq\frac{x-y}{\|x-y\|} \qquad x \neq y \quad \text{(für } x=y \text{ trivial)} \\ \pause
            & \Rightarrow ~\|z\|=1 \quad \Rightarrow ~\|L z\| \leq M \\ \pause
            & \Rightarrow ~\left\|\frac{L(x-y)}{\|x-y\|}\right\| \leq M \quad \Rightarrow ~\|L(x-y)\| \leq M\|x-y\| \\ \pause
            & \Rightarrow ~ \text{gleichmäßig stetig}
        \end{flalign*}
    \end{frame}
%
%    \begin{frame}{In endlichdimensionalem Vektorraum}
%    \end{frame}
%
    \begin{frame}{Beispiel 1}
        Auf den Räumen
        \begin{flalign*}
            & \ell^1\coloneqq \left\{\left(x_n\right)_{n \in \mathbb{N}} ~ \left| ~ \sum_{n=1}^{\infty} \right. |x_n| \text { ist konvergent }\right\}, \quad
            \ell^2\coloneqq \left\{\left(x_n\right)_{n \in \mathbb{N}} ~ \left| ~ \sum_{n=1}^{\infty} \right. \left|x_n\right|^2 \text{ist konvergent} \right\} && \\
            & \text{sind durch} \quad
            \|x\|_1\coloneqq\sum_{n=1}^{\infty}\left|x_n\right| \quad \text {bzw.} \quad\|x\|_2\coloneqq\left(\sum_{n=1}^{\infty}\left|x_n\right|^2\right)^{\frac{1}{2}}
        \end{flalign*}
        Normen erklärt. \hfill
        \begin{flalign*}
            \text{Es sei} \quad L:\left\{\begin{array}{l}
                                             \ell^2 \rightarrow \ell^1, \\
                                             \left(x_1, x_2, x_3, \ldots\right) ~ \mapsto ~ \left(\frac{x_1}{1}, \frac{x_2}{2}, \frac{x_3}{3}, \ldots\right)
            \end{array}\right. &&
        \end{flalign*}
    \end{frame}
%
    \begin{frame}{Beispiel 1: Wohldefiniertheit}
        \noindent
        \begin{flalign*}
            & \text{zu zeigen:} \quad \forall x \in \ell^2 \quad L x \in \ell^1 && \\ \pause
            & \text{sei} \quad a, b \in \ell^2, \quad a b \in \ell^1, \quad (a b)_n\coloneqq a_n b_n \\ \pause
            & \text{Cauchy-Schwarz:} \quad \|a b\|_1 \leq\|a\|_2~\|b\|_2 \\ \pause
            & \text{sei} \quad x \in \ell^2 ~ \text{beliebig}, ~ \quad b=\left(b_n\right)_{n \in \mathbb{N}}, \quad b_n\coloneqq\frac{1}{n} \\
            & \sum_{n=1}^{\infty} \frac{1}{n^2} ~ \text{konvergent} \quad \Rightarrow ~\|b\|_2=\left(\sum_{n=1}^{\infty}\left|b_n\right|^2\right)^{\frac{1}{2}}<\infty \quad \Rightarrow ~ b \in \ell^2 \\
            & \|x b\|_1 \leq \|x\|_2~\|b\|_2 <\infty \quad \Rightarrow ~ x b \in \ell^1 \\ \pause
            & L x=\left(\frac{x_1}{1}, \frac{x_2}{2}, \frac{x_3}{3}, \cdots\right)=x b \pause
            \qquad \Rightarrow ~ L:\left\{\begin{array}{l}
                                              \ell^2 \rightarrow \ell^1 \\ x \mapsto L x\coloneqq b x
            \end{array}\right. ~ \text{definiert eine Abbildung.}
        \end{flalign*}
    \end{frame}
%
    \begin{frame}{Beispiel 1: Linearität}
        für $a, b \in \ell^2$ gilt: \hfill
        \begin{flalign*}
            L(a+b)&=\left(\frac{a_1+b_1}{1}, \frac{a_2+b_2}{2}, \frac{a_3+b_3}{3}, \ldots\right) && \\ \pause
            &=\left(\frac{a_1}{1}, \frac{a_2}{2}, \frac{a_3}{3}, \ldots\right)+\left(\frac{b_1}{1}, \frac{b_2}{2}, \frac{b_3}{3}, \ldots\right)=L a+L b \\ \pause
            L(\lambda a)&=\left(\frac{\lambda a_1}{1}, \frac{\lambda a_2}{2}, \frac{\lambda a_3}{3}, \ldots\right)=\lambda\left(\frac{a_1}{1}, \frac{a_2}{2}, \frac{a_3}{3}, \ldots\right)=\lambda ~ L a
            \qquad \forall a \in \ell^{2}, ~\lambda \in \mathbb{C} \pause
        \end{flalign*}
        $\Rightarrow ~ L$  ist lineare Abbildung
    \end{frame}
%
    \begin{frame}{Beispiel 1: Linearität}
        Darstellung als "$(\infty \times \infty)$-Matrix":
        \begin{flalign*}
            Lx = \left(\begin{array}{llll}
                           b_1 &     &     &        \\
                           & b_2 &     &        \\
                           &     & b_3 &        \\
                           &     &     & \ddots
            \end{array}\right)
            \left(\begin{array}{c}
                      x_1 \\
                      x_2 \\
                      x_3 \\
                      \vdots
            \end{array}\right) =
            \left(\begin{array}{llll}
                      1 &             &             &        \\
                      & \frac{1}{2} &             &        \\
                      &             & \frac{1}{3} &        \\
                      &             &             & \ddots
            \end{array}\right)
            \left(\begin{array}{c}
                      x_1 \\
                      x_2 \\
                      x_3 \\
                      \vdots
            \end{array}\right) &&
        \end{flalign*} \\ \pause
        möglich wegen Wohldefiniertheit $~\Leftrightarrow~$ Reihe konvergiert
    \end{frame}
%
    \begin{frame}{Beispiel 2}
        \noindent
        \begin{flalign*}
            & \text{Es sei} \quad A:\left\{\begin{array}{l}
                                             \ell^{\infty} \rightarrow \ell^{\infty}, \\
                                             \left(x_1, x_2, x_3, \ldots\right) ~ \mapsto ~ \left(\frac{x_1}{1}, \frac{x_2}{2}, \frac{x_3}{3}, \ldots\right)
            \end{array}\right. && \\
            & \text{wobei} ~ \ell^{\infty} \coloneqq \{(x_n)_{n \in \mathbb{N}} \mid \sup|x_n| < \infty\}
        \end{flalign*}
    \end{frame}
%
    \begin{frame}{Beispiel 2: Wohldefiniertheit}
        \noindent
        \begin{flalign*}
            & \text{zu zeigen:} \quad \forall x \in \ell^{\infty} \quad A x \in \ell^{\infty} && \\ \pause
            & \text{sei} \quad x \in \ell^{\infty} \Rightarrow\left|\frac{x_n}{n}\right| \leq\left|x_n\right| \quad \forall n \in \mathbb{N} \\
            & \Rightarrow ~ \|A x\|_{\infty}=\sup _{n \in N}\left|\frac{x_n}{n}\right| \leq \sup \left|x_n\right|=\|x\|_{\infty} \\ \pause
            & x \in \ell^{\infty} \quad \Rightarrow ~ \|A x\|_{\infty} \in \ell^{\infty} \\ \pause
            \vspace{1em}
            & \Rightarrow ~ A:\left\{\begin{array}{l}
                                         \ell^{\infty} \rightarrow \ell^{\infty} \\ x \mapsto A x
            \end{array} \right.~ \text{definiert eine Abbildung.}
        \end{flalign*}
    \end{frame}
%
    \begin{frame}{Beispiel 2: Linearität}
        \noindent
        \begin{flalign*}
            \text{für} ~a, b \in \ell^{\infty}~ \text{gilt:} &&
        \end{flalign*}
        \noindent
        \begin{flalign*}
            A(a+b)&=\left(\frac{a_1+b_1}{1}, \frac{a_2+b_2}{2}, \frac{a_3+b_3}{3}, \ldots\right) && \\ \pause
            &=\left(\frac{a_1}{1}, \frac{a_2}{2}, \frac{a_3}{3}, \ldots\right)+\left(\frac{b_1}{1}, \frac{b_2}{2}, \frac{b_3}{3}, \ldots\right)=A a+A b \\ \pause
            A(\lambda a)&=\left(\frac{\lambda a_1}{1}, \frac{\lambda a_2}{2}, \frac{\lambda a_3}{3}, \ldots\right)=\lambda\left(\frac{a_1}{1}, \frac{a_2}{2}, \frac{a_3}{3}, \ldots\right)=\lambda ~ A a
            \quad \forall a \in \ell^{\infty}, ~\lambda \in \mathbb{C} \pause
        \end{flalign*}
        $\Rightarrow ~ A$  ist lineare Abbildung.
    \end{frame}
%
    \begin{frame}{Beispiel 2: Injektivität}
        \noindent
        \begin{flalign*}
            & \text{zu zeigen:} \quad A x=A y ~ \Rightarrow ~ x=y && \\ \pause
            & A x=A y ~ \Leftrightarrow ~ A(x-y)=0~ \quad \text{wegen Additivität, Homogenität} \\ \pause
            & \text{zu zeigen (vereinfacht):} \quad A x=0 ~ \Rightarrow ~ x=0 \\ \pause
            & \text{sei} ~x \in \ell^{\infty}, \quad A x=0 \\ \pause
            & \Rightarrow ~ \frac{x_n}{n}=0 \quad \forall n \in N \quad \Rightarrow ~ x_n=0 \quad \forall n \in N \quad \Rightarrow ~ x=0 \\ \pause
            & \Rightarrow ~ A ~ \text{ist injektiv.}
        \end{flalign*}
    \end{frame}
%
    \begin{frame}{Beispiel 2: Surjektivität}
        Gegenbeispiel: \\
        $y\coloneqq(1,1,1, \ldots) \in \ell^{\infty}$ \\ \pause
        angenommen, $~ \exists ~ x \in \ell^{\infty}: ~ A x=y$ \\ \pause
        $\Rightarrow ~ \frac{x_n}{n}=y_n, \quad x_n=n y_n=n \quad \forall n \in N$ \\ \pause
        $\Rightarrow ~ x$ unbeschränkt, $~x \notin \ell^{\infty}$ \\ \pause
        $\Rightarrow ~ A$ ist nicht surjektiv. \\ \pause
        $\Rightarrow ~ A$ ist injektiv, aber nicht surjektiv.
    \end{frame}
%
    \begin{frame}{Beispiel 2: Stetigkeit}
        \noindent
        \begin{flalign*}
            & A ~ \text{stetig} ~\Leftrightarrow M\coloneqq\left\{\|A x\|_{\infty}: x \in \ell^{\infty},\|x\|_{\infty} \leq 1\right\}~ \text{beschränkt} && \\ \pause
            & \|A x\|_{\infty} \leq \|x\|_{\infty} ~ \text{siehe oben} ~ \Rightarrow\|A x\|_{\infty} \leq 1 \quad \forall x \in \ell^{\infty}:\left\|x\right\|_{\infty} \leq 1 \\ \pause
            & \Rightarrow M ~ \text{beschränkt,} ~ A ~ \text{stetig.}
        \end{flalign*}
    \end{frame}
%
    \begin{frame}{Beispiel 2: Stetigkeit der Umkehrabbildung}
        \noindent
        \begin{flalign*}
            & x^{(k)}\coloneqq A^{-1} y^{(k)} \quad x_n^{(k)}=\left\{\begin{array}{l}
                                                                         k, \quad \text { falls } n=k \\ 0~ \quad \text { sonst. }
            \end{array}\right. && \\ \pause
            & \text{sei} \quad y^{(k)} \in \ell^{\infty}, \quad k \in \mathbb{N} \quad y_n^{(k)}\coloneqq \begin{cases}
                                                                                                              1, \quad \text { falls } n=k \\ 0~ \quad \text { sonst. }
                                                                                                              &
            \end{cases} \\ \pause
            \vspace{0.5em}
            & \Rightarrow\left\|y^{(k)}\right\|=1 \quad \forall k \in \mathbb{N} \\ \pause
            \vspace{0.5em}
            & \Rightarrow\left\|A^{-1} y^{(k)}\right\|_{\infty}=\left\|x^{(k)}\right\|_{\infty}=k \\ \pause
            & \Rightarrow\left\{\|A^{-1} y\|_{\infty}: y \in X,\| y \|_{\infty} \leq 1\right\} ~ \text{ist unbeschränkt} \\ \pause
            & \Rightarrow A^{-1} ~ \text{ist \emph{nicht} stetig.}
        \end{flalign*}
    \end{frame}
%
    \begin{frame}{Beispiel 3}
        Gegeben ist die lineare Abbildung
        \begin{flalign*}
            \frac{\mathrm{d}}{\mathrm{d} X}:\left\{\begin{array}{l}
                                                       \mathbb{R}[X] \rightarrow \mathbb{R}[X], \\
                                                       p ~ \mapsto ~ \frac{\mathrm{d}}{\mathrm{d} X} p,
            \end{array}\right. &&
        \end{flalign*}
        die jedem Polynom seine Ableitung zuweist nach dem Schema: \hfill
        \begin{flalign*}
            \frac{\mathrm{d}}{\mathrm{d} X}\left(\sum_{j=0}^n a_n X^n\right)=\sum_{j=1}^n n a_n X^{n-1} &&
        \end{flalign*}
    \end{frame}
%
    \begin{frame}{Beispiel 3: Linearität}
        \noindent
        \begin{flalign*}
            \forall ~ \lambda \in \mathbb{R}, \quad p,~ q \in \mathbb{R}[X] ~ \text{gilt:} \qquad
            \frac{\mathrm{d}}{\mathrm{d} X}(\lambda p+q)=\lambda \frac{\mathrm{d}}{\mathrm{d} X} p+\frac{\mathrm{d}}{\mathrm{d} X} q. &&
        \end{flalign*}
    \end{frame}
%
    \begin{frame}{Beispiel 3: Injektivität, Surjektivität}
        \textbf{Injektivität:} \\
        Gegenbeispiel: \\
        $\frac{\mathrm{d}}{\mathrm{d} X}(X+1)=1=\frac{\mathrm{d}}{\mathrm{d} X}(X-1)$ \\ \pause
        \vspace{1em}
        \textbf{Surjektivität:} \\
        $\forall ~ p=a_0+a_1 X+\cdots+a_n X^n ~ \in ~ \mathbb{R}[X]$ \\
        $\exists ~ P=a_0 X+\frac{1}{2} a_1 X^2+\cdots+\frac{1}{n+1} a_n X^{n+1} ~ \in ~ \mathbb{R}[X]: ~ \frac{\mathrm{d}}{\mathrm{d} X} (P) = p$. \\ \pause
        $\Rightarrow ~ \frac{\mathrm{d}}{\mathrm{d} X}$ ist surjektiv, aber nicht injektiv.
    \end{frame}
%
    \begin{frame}{Lässt sich jede lineare Abbildung als Produkt mit einer Matrix schreiben?}
        \noindent
        \begin{flalign*}
            \left(\begin{array}{cccc}
                      a_{11} & a_{12} & a_{13} & \cdots \\
                      a_{21} & a_{22} & a_{23} &        \\
                      a_{31} & a_{32} & a_{33} &        \\
                      \vdots &        &        & \ddots
            \end{array}\right)\left(\begin{array}{c}
                                        x_1 \\
                                        x_2 \\
                                        x_3 \\
                                        \vdots
            \end{array}\right) \pause =\left(\begin{array}{c}
                                                 \sum_{n=1}^{\infty} a_{1 n} x_n \\
                                                 \sum_{n=1}^{\infty} a_{2 n} x_n \\
                                                 \sum_{n=1}^{\infty} a_{3 n} x_n \\
                                                 \vdots
            \end{array}\right) &&
        \end{flalign*}
    \end{frame}
%\AERbeamerSetFooterText{References}%
%\begin{frame}[allowframebreaks]{References}%
%    \printbibliography[heading=none]%
%\end{frame}%
%
% End with titlepage
%    \AERbeamerTitlePageDefault%
%
\end{document}%
%
%
